The selection bias in estimates of mortality change among the less educated is widely recognized and has been examined in other studies. \citet{Meara2008}, \citet{Bound2015}, \citet{Hendi2015}, and \citet{Leive2020} (henceforth MBHL) adjust for this bias by randomly reassigning deaths to different education bins, so that bin sizes are comparable across time. For example, to obtain an estimate of mortality in the bottom quartile of the education distribution in 1992 when 19\% of 50-year-old men are dropouts and an additional 36\% are high school graduates, they would reassign 6/36 = 16.6\% of the high school graduate population and 16.6\% of high school graduate deaths to the bottom bin.

This approach is equivalent to assuming that the conditional expectation function of mortality given latent education rank takes a specific functional form---a step function that is totally flat in each education category, and has a discrete jump at each education boundary. To be concrete, under this assumption, an individual who just barely managed to complete high school (and thus has the lowest latent education rank among high school graduates) has exactly the same implied socioeconomic status and expected mortality risk as a high school graduate who was right at the margin of completing a two year college degree (and thus has the highest latent education rank among high school graduates). Standard human capital theory suggests that the true functional form is not flat in each category---the high school educated individuals who were at the margin of completing some higher education would have had higher socioeconomic status than those who barely made it to high school, and thus lower mortality risk.

This implicit functional form is nevertheless considered a valid functional form in our bounding exercise, which allows arbitrary steps and slope changes at education boundaries. But this functional form underestimates mortality among the least educated in all periods, because it constructs bins by combining dropouts with average high school graduates -- even though the high school graduates with the lowest latent ranks are likely to have higher mortality than average high school graduates.  The downward bias on mortality among the least educated will be the highest when the education-mortality gradient is steep. Because this gradient has steepened over time \citep{Goldring2016}, the downward bias on mortality is higher in 2018 than in 1992, which means that mortality change among the least educated is biased downward when we use this functional form.

Note finally that these other approaches are likely to be increasingly biased when the bin boundary shifts more over time, or when the desired outcome percentiles are very different from the bin boundaries in the raw data, but the estimates from the MBHL function will not reflect this source of error. One reason that few of these authors focus on the very bottom of the education distribution (or the percentiles approximating high school dropouts) is that the large population change in dropouts (among women, from 19\% of the population in 1992 to 8\% in 2018) leads to a substantial potential for bias.  With our approach, in contrast, the bounds reflect the uncertainty in the estimates and become wider in cases like these where bin boundaries have shifted substantially. Our bounds thus accurately convey the uncertainty due to misalignment between desired outcome percentiles and bin boundaries in the data.

The MBHL function generates a mortality estimate among the least educated that is close to our lower bound mortality estimate. Our results are therefore entirely consistent with \citet{Meara2008}, who find that death rates among those with a high school education or less are diverging from those with any college education. We find similar effects for the period up to 2000 studied by \citet{Meara2008}, and show that (i) mortality by education continues to diverge from 2000--2018; and (ii) the bottom 10\% of whites do particularly badly in both periods.

In contrast, \citet{Bound2015} argue that the composition adjustment effectively erases large mortality increases among non-Hispanic whites in the bottom 25\%, though they continue to find average decreases in life expectancy at age 25 among non-Hispanic whites in this education group.  These differences can be reconciled with our finding of substantially rising mortality rates among the least educated non-Hispanic whites. First, as noted in this section, the Bound et al. estimates are at the lower bound of mortality change because of the implicit functional form assumption. Second, we find that mortality increases are most severe among the bottom \textit{10\%}; extending the interval to the bottom 25\% substantially attenuates the estimated mortality change, and makes the functional form bias larger. Third, we focus on middle-age mortality change, because of known problems of age inaccuracy among older ages \citep{Olshansky2012}; trends among individuals aged 70 and older may substantially influence life expectancy and be different from those studied here.

Hendi (2015, 2017) use data from NHIS to argue that mortality is rising for white women without high school education but not white men. We find worse outcomes for both women and men because: (i) the random reassignment approach of Hendi (2015, 2017) biases downward estimates of mortality change; (ii) as noted in detail by Sasson (2017), the NHIS consistently samples a healthier population than that reflected in population vital statistics, and the mortality followup sample sizes are too small to precisely estimate mortality changes for less educated groups. We perform a similar analysis in Appendix~\ref{sec:app_nhis}, showing that while NHIS generates point estimates of lower mortality changes for less educated women (but not men), the estimates are extremely noisy and our estimates are well within the 95\% confidence intervals of the NHIS measures. In some age/education groups, NHIS mortality changes are estimated from just a handful of deaths.

Hendi (2015, 2017) also raise the possibility that over-reporting of education in death statistics may have changed over time, as noted by \citet{Sorlie1996}, leading to underestimates of death rates among older cohorts of dropouts. While we cannot rule out this form of bias, we present a range of evidence in Appendix~\ref{sec:app_robust} (summarized in Section~\ref{sec:robust}) suggesting that misreporting of education in death records cannot explain larges increases in mortality among white men and women.
