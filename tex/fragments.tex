------------------------------
infinite support - nov 2020 
------------------------------
However, $F$ need not have finite
support. For all $F$, $r_k$ is finite for all $k \in
\{0,\dots,K\}$. 

In the case where $\underline{y} = -\infty$ has infinite support, then
  $x_1^* = x_1 = 0$. In the case where $\overline{y} = \infty$, $x_{K+1}^*
  = x_{K+1} = 100$.

--------------------------
curvature - nov 2020 
--------------------------

This subsection explains how we obtain a benchmark for the curvature constraint
using the data from \citet{Chetty2016b}. \citet{Chetty2016b} provide
data on mortality rates for U.S. men and women above age 40 from years
2001--2014. We collapse the data to three-year periods and five-year
age groups.\footnote{Since the number of years is not divisible by 3,
  we group years 2001 and 2002.}\footnote{Because people are ranked
  within the percentile for their own age, gender, and year, this
  departs slightly from the ranking procedure we use in the body.} We then fit fifth-order
polynomials to the mortality-income percentile data we observe. 

We construct a $\overline{C}$ that holds for all mortality-income
functions as follows. Using the estimated polynomial fit, we
analytically compute the second derivative at every value for every
polynomial function. 

Across
all age and years, the absolute value o second derivative of these polynomials
that we observe at any point, divided by the mean mortality across all
percentiles, is approximately 2\%. Hence we choose a conservative
curvature constraint of 3\%. Because polynomial fits can be inaccurate
near the tails, we also compute the largest second-derivative
(normalized by the mean mortality across all percentiles) within the
set of percentiles $[5,95]$. This value is 1.5\%. 


non-mon robust graph removed from Figure D2, nov. 2020
------------------------------------------------------
\vspace{-.6cm} Panel C: No Monotonicity Assumption
  \end{center}
%\vspace{-1.4cm}
  \begin{center}
    \includegraphics[scale=0.78]{\mortalitypath/causes-nomon-2-1} &
  \end{center}
  \begin{center}


app_comparison nov. 2020
-------------------------
; the bounds in Panel D of Appendix Figure~\ref{fig:robust} contain the mortality estimates generated by the Meara, Richards and Cutler / Bound et al. (MRCB) function. 

CURVATURE APPENDIX NOV. 2020
----------------------------
The following constrained optimization problem provides the lower bound on mortality in the bottom 10\% of the latent education distribution, or $Mortality_{0-10}^{min}$: \spacing{1}
\begin{align}
  \label{eq:opt2}
  Mortality_{0-10}^{min} &= \underset{ \hat{\gamma} \in [0,100000]^{100} }{ \text{min} } \frac{1}{10} \sum_{x=1}^{10} (\hat{\gamma}_x) \\
  &\nonumber \\
  &\nonumber \text{such that} \\
  &\nonumber \\
  \tag{Monotonicity} \hat{\gamma_x} \text{ is weakly decreasing in } x \\
  &\nonumber \\
  \nonumber \lvert (\hat{\gamma}_{x+1} - \hat{\gamma}_{x}) - (\hat{\gamma}_{x} - \hat{\gamma}_{x-1})\rvert \leq \overline{C}, \\
  \tag{Curvature} \text{unless there is an education level boundary between $x-1$ and $x+1$} \\
  &\nonumber \\
  \tag{MSE Minimization} \sum_{k=1}^4 \left[ \frac{\Vert X_k \Vert}{100} \left( \left( \frac{1}{\Vert X_k \Vert} \sum_{x \in X_k} \hat{\gamma}_x \right) - \overline{r}_k \right)^2 \right] &= \underbar{MSE}\text{.}
\end{align}
\spacing{1.5}
\noindent
The minimand is the average mortality in the bottom 10\%. $\overline{C}$ is the highest allowable curvature across any three points that do not span an education boundary; this expression prevents sharp kinks or breaks in the mortality function. The four education levels observed in the data are indexed by $k$.  $X_k$ is the set of integer education ranks in bin $k$ and $\Vert X_k \Vert$ is the width of bin $k$, or the number of integer ranks in that bin. $\overline{r}_k$ is the observed mortality in education bin $k$, and \underbar{MSE} is the lowest mean-squared error obtainable out of the entire set of education-mortality functions, which is typically zero.  The complementary maximization problem obtains the upper bound on mortality in the bottom 10\%.



CONCLUSION Nov. 2020
---------------------
The magnitudes exceed that of the Russian mortality crisis, where male and female mortality grew by 18\% (men) and 29\% (women) between 1991--2001 (\hl{citep }, Table 3).
\begin{comment}
https://www.ncbi.nlm.nih.gov/pmc/articles/PMC259165/#:~:text=Results%20Mortality%20increased%20substantially%20after,years%20among%20women%20by%202001.&text=An%20extra%202.5%2D3%20million,expected%20based%20on%201991%20mortality.
\end{comment}
Of course, the mortality increases we find in the U.S. are concentrated within one part of the education distribution, rather than the entire population in the country.


METHODS November 2020
---------------------
In that case, $$ r_k = \frac{1}{x_{k+1} - x_k} \int_{x_k}^{x_{k+1}}
Y(x)dx,$$ substituting the probability distribution function for the
uniform distribution within bin $k$. Then we derive the following
proposition. 

Proof intuition:
--
The upper and lower bounds of the
expression are somewhat involved,
but the logic is simple. For any rank, we consider the best- and
worst-case survival rate that can satisfy the observed bin means
and is monotonic. For point $x$ that is just slightly larger than
$x_k$, there exists a CEF that is flat in the interval $[x_k,x]$. This
CEF still has a mean in bin $k$ equal to $r_k$. However, for points $x >
x_k^*$, it is not possible that the CEF is flat up to that
point. Intuitively, if the CEF's bin mean is equal to $r_k$ but the value of
the CEF is flat up to $x$, the values $x' >
x$ must exceed the bin mean in the \textit{subsequent} bin --- a
violation of monotonicity. Thus, $x_k^*$ defines a natural cut-off point beyond which the lower bound
of the CEF must exceed $r_k$. 

An advantage of our proposition is that
the bounds are explicitly defined with very little data (just the bin
means and share in each bin). Moreover, because they rely on only elementary
arithmetic, the bounds can be computed instantaneously by any
statistical software.

We focus on women aged 50--54, because this is a group whose education composition has shifted substantially over time.

signpost from top of results section:
----
We begin by noting details about applying the methodology in Section
\ref{sec:method} to our particular mortality setting. 


RESULTS TO MAKE:
- demonstration graph with two examples
- CEF graph with Manski-Tamer bounds, NRA bounds, and maybe curvature bounds.
  - Panel A: MT and NRA.  Panel B: Curv0.1 and Curv0.2
- Table app proofs (maybe from bounds paper) showing bounds under different scenarios
- Fix bias figure so period 1 has collapsed bounds.
- Do we have the curvature spline section?

JAMA edits
----------

potential changes:
- cut Figure 2-B?
- put back Chetty problems with income paragraph
- cut Ruhm cite to working paper?

The SM contain the unadjusted estimates and
detailed time series for all ages and subgroups.

Accidents include deaths from falls,
motor vehicle accidents, and assault. Deaths in the ``other'' category
are broadly distributed across a range of causes, the largest of which
are deaths from respiratory, endocrine, nutritional and metabolic
diseases (Appendix Table~\ref{tab:icd_causes}).  

An example provides intuition for the method. If mean mortality in the
bottom 20\% is 2000 deaths per 100,000, and mean mortality in
percentiles 20--50 is 1000 deaths per 100,000, then the monotonicity
assumption requires that mean mortality in the bottom 10\% cannot be
higher than 3000 deaths per 100,000. Any higher mortality rate among
the bottom 10\% would require a mortality rate in percentiles 10-20
that is \textit{higher} than the mean in percentiles 20-50. This would
mean that mortality must \textit{rise} as one crosses the 20th
percentile --- violating the assumption of monotonicity.

\footnote{To make the normalization concrete, consider
  50--54-year-old white men in the bottom 10\% of the education
  distribution. The midpoint estimate of deaths of despair increased
  from 93 deaths per 100,000 to 336 deaths per 100,000, a 261\%
  increase. The midpoint estimate of deaths from all causes from this
  group was 1347 per 100,000 in 1992--1994, so the change in deaths
  from despair alone would have caused the total mortality rate to
  rise by 18\% over this period---this is the number that we plot on
  the figure next to the change in total mortality. Total mortality
  increased by between 34\% and 63\% for this group, so deaths from
  despair account for between half to a third of the all-cause
  mortality increase.}

Deaths
from accidents are also rising, but like deaths of despair for this
group, the base level is very low, so they do not account for a large
change in total mortality.


The mortality divergence across education
groups cannot be explained only by changes in deaths from despair.

LAST LINE OF RESULTS:  These may be useful for
researchers interested in further exploring the drivers of divergent
mortality by education.

GOOD SENTENCES TO PUT BACK INTO DISCUSSION: There are also several
data advantages to focusing on education. First, current income-based
estimates exclude individuals with zero income, who are presumably
negatively selected in terms of mortality; this exclusion could also
explain some of the estimated mortality differences between the
poorest and the least educated. Second, income estimates based on tax
records cannot be directly linked to race, which this study shows is
an important factor in understanding recent mortality changes.


It is difficult to forecast future changes in mortality from current
trends. Given recent evidence that inequality in infant health is
declining \cite{Currie2018}, we can hope that mortality among the
least educated whites will not continue to rise at the current rate of
3\% per year. However, the trends described in this paper show little
sign of slowing; the negative trend among the least educated continues
into the 1980-84 birth cohort and in fact extends deeper into the
education distribution for later-born cohorts. Better understanding the
causes of mortality divergence in the United States should remain high
on the research agenda.



cut from results summary in intro
---------------------------------
\footnote{This
  is approximately the share of dropouts in 2015.} Among whites in the
next percentile bin, men of all ages and older women experience
comparatively minor mortality changes, though white under 45 experience
substantially higher risk of death up to the 45th percentile of the
education distribution. Substantial improvements in survival occur
only among whites in the top 30\%.

cut from results
----------------
Accidents make up a small share of total mortality except
among the youngest age groups.




MAIN FINDINGS
--------------
We highlight three main findings from this approach. First, the rising
21st century mortality of white Americans is driven almost entirely by
the bottom 10\% of the education distribution. Among 50-year-old
whites, mortality among the bottom 10\% has risen dramatically, by
between \hl{X\% and Y\%}. Because this is a constant percentile group,
compositional changes in the educational distribution cannot
mechanically explain this. Once compositional changes are taken into
account, mortality is flat for the next highest group (percentiles
10-45, corresponding to those with high school degrees in 2015). The
only group with sustained large mortality declines from 1992 is the
most educated 30\%. The distribution of mortality change is similar
for all middle-aged whites.

Second, we find the same mortality divergence for the least educated
blacks, though on a different average trend. For instance, mortality
has fallen slightly for blacks in the bottom 10\% (from a very high
baseline level, especially among men), while it has declined
substantially for the top 90\%.  We do not look separately at trends
among Hispanics because the higher in- and out-migration of this group
makes it more difficult to interpret trends in total
mortality.

Third, we find that mortality increases among the least educated 10\%
of whites are driven by increases in mortality from all causes, not
only from the deaths from suicide, poisoning and liver disease noted
cby Case and Deaton \cite{Case2015,Case2017}. Among 50-year-old white
women, death rate from heart disease, cancer, and many other diseases
are also on the rise.

 Finally, we
have focused on a single age group of 50--54 year old individuals
because this group has been the focus of recent high profile work,
most notably by Case and Deaton \cite{Case2015,Case2017}. 

We follow the literature in limiting our
analysis to groups under the age of 70 at the time of death because
death records are considered less reliable for the study of this group.



DISCUSSION
------------
The rising mortality of the least educated has been noted in the past,
most prominently by Olshansky \cite{Olshansky2012}. However, the
dramatic educational compositional changes that have occurred among
middle-aged Americans from the 1990s to the present has cast doubt on
whether mortality is indeed rising in the bottom 10\%, or whether the
entire effect among dropouts is driven by compositional
changes. Indeed, work since Olshansky has explicitly avoided looking
separately at high school dropouts because of these compositional
changes, and has focused on the larger set of people with high school
or less \cite{Cutler2011,Case2015,Case2017}. Because this set of
individuals is larger, the bias due to compositional changes is likely
to be smaller. Our work shows that this approach in fact draws
attention away from the group where health outcomes are deteriorating
the most rapidly: those at the very bottom of the education
distribution.

Our focus is on describing mortality trends across the educational
distribution, holding groups constant. Explaining the causes of
mortality increases at the very bottom of the education distribution
is beyond the scope of this paper. However, the identification of the
population subgroups who are driving adverse mortality trends will be
central to diagnosing these mortality increases in further research.



Say something like ``across all groups, deaths of despair account for
10\% of total deaths in 2015 but for 40\% of the mortality increase
since 1992.''  Except this is a case/deaton point, not our point.  Or
rather, is the point that they explain only a small share of the total increase?


cut from methods
----------------
An alternative approach is to assume that mortality is weakly
monotonically decreasing in the latent educational rank. This is a
common assumption in the literature, and it holds very strongly across
all ventiles of the income rank / mortality relationship
\cite{Chetty2016b}. However, this method is not available for older
cohorts of women in the 1990s, among whom those with high school
education occasionally have marginally higher mortality than high
school dropouts, and in some cases those with bachelor's degrees have
higher mortality than those with only two years of college. We can
estimate constant rank mortality differences for almost all cohorts
from 2000 to present under the assumption of monotonicity, without a
curvature constraint; we find very similar bounds and point estimates
to our primary curvature constrained estimates. While we suspect that
the non-monotonicity in earlier cohorts is due to random noise in
the mortality data, interpreting it at face value only strengthens
our conclusion that the bottom 10\% of the education distribution has
diverged dramatically from the top 90\% since 2000.


Appendix Figure XX shows scatterplots of mortality from deaths that
are due to chronic liver disease, poisoning and suicide, denoted
``deaths of despair'' by Case and Deaton
\cite{Case2015,Case2017}. These deaths are increasing for all
races, genders and education groups with the exception of black men
with some college or higher; the changes are again the greatest for
high school dropouts.



old cohort plot results
-----------------------
In Figure~\ref{fig:cohort}, we present another perspective on estimates of mortality
change for age groups from 25 to 65. Each series shows estimates of
total mortality for a given birth cohort at each age from 1992 to
2015. Birth cohorts are in five-year bins; every second birth cohort
is omitted for clarity of presentation.\footnote{Results are similar
  if we instead use ten-year birth cohorts, but the latter would be
  subject to the critique of Gelman and Auerbach \cite{Gelman2016}.}
Each point shows the average age and mortality for a given birth
cohort in a given year.  The green line, for instance, shows mortality
for the 1960-1964 birth. We observe this birth cohort from 1992, when
the average age is 32, to 2015, when the average age is 57. Each
series is rising because mortality increases with age. Changes in
mortality rates over time can be seen in the vertical differences
between the different series. As above, education groups are divided
according to constant rank percentiles, so none of these mortality
changes can be explained by the change size or relative rank of any
given education group. ANR bounds are less precise (indicated by
thicker lines) for cohorts where the
education distribution differs substantially from the percentile bin
boundaries that we use.

We focus first on white women. The graph shows that in the bottom 10\%
of the education distribution, mortality has been rising steadily
across all cohorts and ages, from the 1940-44 birth cohort to the
1980-84 birth cohort. This confirms that the results above are not
limited to 50--54 year olds. In percentiles 10--45, mortality has been
approximately unchanged until the 1960-64 cohort, at which point it
begins to rise. For percentiles 45-70, mortality declines from the
1930s to the 1950s, and then flattens. Among white women, the top 30\%
in the education distribution are the only group to experience
consistent mortality improvements across the entire sample period.

The cohort plot for white men can be similarly characterized, with
mortality rising on average across all cohorts for the bottom 10\%,
and declining across all cohorts for the top 30\%. The intermediate
groups show marginal declines in mortality from the 1930s to the 1970s
cohorts; the 1980-84 cohort of white men is something of an outlier,
with 30-40\% higher mortality than the cohort ten years earlier.
The cohort plots for black women and black men similarly show that the
divergent path of mortality for the least educated 10\% largely holds
across all age groups and birth cohorts. 

Supplementary Table~\ref{tab:all_mort} presents bounds on mortality
change from 1992--2015 for all 5-year age bins from ages 35--64, for
each of these constant percentile education groups, by race and
gender.

\subsubsection{Composition-Adjusted Changes in Mortality by Cause of Death}

This section expands the analysis to examine the causes of death
underlying the mortality changes described
above. Table~\ref{tab:causes} presents bounds on mortality change from
1992--2015 for all race, gender and education percentile groups, by
cause of death.

The trends in total mortality do not reflect only a change in the pattern of deaths from suicide,
poisoning and chronic liver disease, but rather reflect changes in a
broad range of causes of death. Deaths of despair stand out for their
proportional change in part because they begin at such a low level in
1992. But for white women, mortality is also increasing from heart
disease, cancer, accidents and other causes (which are predominantly
other diseases) among the bottom 10\% of the education
distribution. Among middle-aged white men in the bottom 10\%, there
are substantial increases in deaths from accidents, other diseases and
from cancer; these changes are concentrated among men over 50. Among
the least educated black men and women, mortality is declining on
average in most categories, but the divergence of the bottom 10\%
holds clearly in all categories.
