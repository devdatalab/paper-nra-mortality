\begin{table}[H]
  \caption{Bounds on Mortality Thoughout the Education Rank Distribution \cnewline 50--54-Year-Old Women, All Races}
  \label{tab:bound_stats}
  \begin{center}
    \begin{tabular}{c}
      \panel{A. 1992--1994} \\

      \begin{tabular}{lccc}
\hline
Statistic                                    & Monotonicity Only                                 & Curvature Only                                            & Monotonicity and                                      \\
                                             & $(\overline{C}=\infty)$                           & $(\overline{C}=3)$                                        & Curvature $\overline{C}=3$                            \\
\hline
$Y(x=10)$: First Quintile Median              & [455.9, 682.1]   & [343.3, 793.8]   & [456.1, 614.7]   \\
$Y(x=25)$: Bottom Half Median                 & [427.7, 587.2]   & [0.0, 1163.1]   & [436.9, 586.7]   \\
$Y(x=8)$: Median $\le$ High School (1992--94)   & [455.9, 738.0]   & [453.1, 726.2]   & [485.9, 638.8]   \\
$Y(x=4)$: Median $\le$ High School (2016--18)   & [455.9, 1013.2]   & [263.6, 972.2]   & [573.1, 730.5]   \\
\rule{0pt}{2ex}                              &                                                   &                                                           &                                                       \\
$\mu_0^{20}$: First Quintile Mean            & [570.2, 587.2] & [539.0, 607.1] & [567.6, 586.2] \\
$\mu_0^{50}$: Bottom Half Mean               & [501.6, 530.7] & [431.3, 582.1] & [504.3, 529.5] \\
$\mu_0^{16}$: Mean $\le$ High School (1992--94) & [587.2, 598.7] & [585.3, 595.2] & [588.1, 595.1] \\
$\mu_0^{8}$: Mean $\le$ High School (2016--18) & [587.2, 741.5] & [259.7, 1041.2] & [587.5, 725.6] \\
\hline
\end{tabular}

 \\
      \\
      
      \panel{B. 2016--2018} \\
      
      \begin{tabular}{lccc}
\hline
Statistic                                    & Monotonicity Only                                 & Curvature Only                                            & Monotonicity and                                      \\
                                             & $(\overline{C}=\infty)$                           & $(\overline{C}=3)$                                        & Curvature $\overline{C}=3$                            \\
\hline
$Y(x=10)$: First Quintile Median              & [516.0, 799.9]   & [284.9, 1074.7]   & [534.3, 799.8]   \\
$Y(x=25)$: Bottom Half Median                 & [318.5, 685.4]   & [208.2, 775.5]   & [349.0, 600.5]   \\
$Y(x=8)$: Median $\le$ High School (1992--94)   & [535.3, 799.9]   & [417.0, 1009.8]   & [535.4, 799.8]   \\
$Y(x=4)$: Median $\le$ High School (2016--18)   & [535.3, 1046.3]   & [737.3, 831.1]   & [733.8, 816.3]   \\
\rule{0pt}{2ex}                              &                                                   &                                                           &                                                       \\
$\mu_0^{20}$: First Quintile Mean            & [640.1, 799.9] & [476.9, 903.0] & [641.2, 783.0] \\
$\mu_0^{50}$: Bottom Half Mean               & [520.8, 570.1] & [455.5, 553.0] & [521.3, 551.2] \\
$\mu_0^{16}$: Mean $\le$ High School (1992--94) & [666.2, 799.9] & [551.7, 952.2] & [667.7, 793.0] \\
$\mu_0^{8}$: Mean $\le$ High School (2016--18) & [797.2, 799.9] & [799.9, 799.9] & [799.9, 799.9] \\
\hline
\end{tabular}

 \\
      \hline
    \end{tabular}
\end{center}
\end{table}
\footnotesize{Note: ``White'' refers to non-Hispanic white and
  ``black'' refers to non-Hispanic black. The table shows bounds on
  mortality in 1992--94 (Panel A) and 2016--18 (Panel B) at various
  ranks or rank ranges in the education distribution. The notation $Y(x=i)=E(Y|x=i)$
  describes mortality at education percentile $i$, and $\mu_a^b$
  describes average mortality between education percentiles $a$ and
  $b$. $\overline{C}$ is the maximum percentage change in mortality
  function curvature allowed in any one percentile that does not
  correspond to an education bin boundary. Sources: ACS, CPS, and NCHS}


\begin{table}[H]
  \caption{Age-Adjusted Changes in All-Cause Mortality \cnewline by Education Percentile, 1992--94 to 2016--18}
  \label{tab:mort_changes}
  \begin{center}
    \begin{tabular}{lcccc}
  \hline
              & \multicolumn{4}{p}{Education Percentile Group} \\
  \hline
              & 0--10th    & 10th--45th & 45th--70th & 70th--100th \\
  \hline
  White Women & (+77\%, +111\%) & (-0\%, +21\%) & (-39\%, -4\%) & (-48\%, -39\%)  \\
  White Men   & (+50\%, +68\%) & (-6\%, +5\%) & (-40\%, -18\%) & (-52\%, -45\%)  \\
  Black Women & (+11\%, +17\%) & (-28\%, -21\%) & (-47\%, -33\%) & (-55\%, -51\%)  \\
  Black Men   & (-0\%, +3\%) & (-33\%, -29\%) & (-54\%, -44\%) & (-67\%, -64\%)  \\
  
  \hline
\end{tabular}

\end{center}
\end{table}
\footnotesize{Note: ``White'' refers to non-Hispanic white and
  ``black'' refers to non-Hispanic black. The table shows the percent
  change in all-cause mortality, defined as total deaths in a year
  divided by population. To hold the population distribution constant,
  we weight the age-specific mortality rates from the data with the
  standardized U.S. population distribution for ages 25--69. We use
  age-specific mortality rates from each period, but a single set of
  weights for all periods.}
