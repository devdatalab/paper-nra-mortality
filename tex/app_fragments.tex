\begin{landscape}
%%%%%%%%%%%%%%%%%%%%%%%%%%%%%%
%% SCATTERS- 30, 40, 60, 70 %%
%%%%%%%%%%%%%%%%%%%%%%%%%%%%%%
\foreach \n in {3,4,6,7}{
  
  \begin{figure}[H]
    \caption{Total Mortality by Education Group Over Time, Age \n0-\n4}
    \label{fig:mort_scatters_app}
    \begin{center}
  
      \includegraphics[scale=1.2]{\mortdescpath/scatter_tmort_\n0}
        \hline
  
  
    \end{center}
    \noindent
    \footnotesize{The figure shows a scatterplot of total mortality
      over time, for ages \n0-\n4.}
  \end{figure}
}

%%%%%%%%%%%%%%%%%%%%%%%%%%%%%%%%%
%% TRENDS -- DEATHS OF DESPAIR %%
%%%%%%%%%%%%%%%%%%%%%%%%%%%%%%%%%

\begin{figure}[H]
  \caption{Constant Composition Mortality Estimates, Ages 50--54,
    1992-2015 \cnewline Deaths of Despair}
  \label{fig:trend_despair_nomon}
  \begin{center}

    \includegraphics[scale=1.2]{\mortdescpath/new_joint_50_d_nomon_3}
      \hline

  \end{center}
  \noindent
  \footnotesize{The figure shows total mortality for age group 50-54,
    divided into population subgroups who reflect constant sizes and
    rank positions in the education distribution. The rank positions
    are determined by the population distribution of education in
    1992--the graph therefore shows changing mortality for the
    subgroups of the population based on 1992 education levels, with
    those rank positions held constant.}

\end{figure}
\end{landscape}
%%%%%%%%%%%%%%%%%%%%%%%%%%%%%%
%% TRENDS -- 30, 40, 60, 70 %%
%%%%%%%%%%%%%%%%%%%%%%%%%%%%%%

%%%%%%%%%%%%%%%%%%%%%%%%%%%%%%%
%% Table: Mortality Changes  %%
%%%%%%%%%%%%%%%%%%%%%%%%%%%%%%%
\begin{table}[H]
\caption{Change in Total Mortality, 1992--2015}
\label{tab:all_mort}
\begin{center}
\textbf{Panel A: Non-Hispanic White Women} 
\end{center}
\begin{center}
\small{\input{\mortdescpath/table_paper_changes_2_1_tmon-step}} 
\end{center}
\begin{center}
\textbf{Panel B: Non-Hispanic White Men} 
\end{center}
\begin{center}
\small{\input{\mortdescpath/table_paper_changes_1_1_tmon-step}}
\end{center} 
\begin{center}
\textbf{Panel C: Non-Hispanic Black Women} 
\end{center}
\begin{center}
\small{\input{\mortdescpath/table_paper_changes_2_2_tmon-step}}
\end{center}
\begin{center}
\textbf{Panel D: Non-Hispanic Black Men} 
\end{center}
\begin{center}
\small{\input{\mortdescpath/table_paper_changes_1_2_tmon-step}}
\end{center}
\end{table}



%%%%%%%%%%%%%%%%%%%%%%%
%% hispanic identity %%
%%%%%%%%%%%%%%%%%%%%%%%

Other researchers have examined in detail whether Hispanic identity is
reported differentially between death records and sample surveys, and
have found these two data sources to be reliable and consistent on the
question of Hispanic identity .
