This section provides additional details on data construction.  The
All Cause Mortality file provided by the National Center for Health
Statistics reports the number of deaths, by age, race, gender,
education and state.

\textbf{Mortality data.} Beginning in 2003, a substantial share of states only report
educations in coarse bins. The bins are: 
\begin{itemize}
\item 8th grade or less
\item 9th - 12th grade, no diploma
\item High school graduate or GED 
\item Some college, but no degree
\item Associate degree
\item Bachelor's degree
\item Master's degree
\item Doctorate or professional degree
\item Unknown 
\end{itemize}

To reduce noise, we slightly coarsen the bins. We put the small share of people
people with an 8th grade or less into the HS Dropouts group. We aggregate HS dropouts and
middle-school dropouts due to concerns about statistical precision.

We aggregate some college and Associate degrees into the Some College
group; it is not clear which group would have
higher rank or socioeconomic status on average. To reduce noise, we aggregate
Bachelor's, Master's, and Doctorate/professional degrees. These
aggregation choices are unlikely to affect our estimates of mortality
at the bottom of the distribution. 

We exclude deaths of foreign residents. 

\textbf{Missing education.} Georgia, Oklahoma, Rhode
Island, and South Carolina do not consistently report education data
linked to the death certificates. Because
their entry and exit from the data could bias mortality trends, we
drop mortality records and population totals for these states. The
mortality rates we report are thus mortality rates for the remaining states. 

The remaining mortality records occasionally report missing
education. In each age-gender-race-year category, we obtain the
proportion of death records with non-missing educations belonging to
each education group. We assign an education group to mortality
records where education is missing, assuming that the missing
distribution is the same as the non-missing distribution. For example,
if 25\% of mortality records with non-missing education in a
given cell have a high school degree only, we assign 25\% of the
mortality records with missing education to have a high school
degree. This practice is standard; see Case and Deaton
(\citeyear{Case2015,Case2017}).

Missing data cannot account for the large mortality changes
described in the body of the paper. After dropping the four states, 2.90\% of
white and 4.34\% of black mortality records are missing
education (across all years). Roughly 20\% of missing deaths among
whites (and 25\% among blacks) are assigned to high school
dropouts; even in the extreme case where \textit{all} of these
assignments were incorrect, it would erroneously assign only about 0.6\%
of white and 1\% of black deaths to the bottom education bin, thus creating very
little bias.\footnote{By 2000 and after, the share of missing data falls by
  about a third. Thus, the trends over the past 18 years are even less subject to this
  concern.} 

\textbf{CPS.} Following \cite{Case2017}, we use the March
CPS extracts prepared by the Center for Economic and Policy
Research. 

\textbf{Harmonizing educations.} A standard issue when harmonizing
educational attainment data across several datasets is the treatment
of people who drop out of high school in 12th grade. \citet{Jaeger1997}
recommends treating 12th grade dropouts as people who complete
high school, since in earlier datasets, people who complete part of
12th grade were typically treated as people who completed
high-school. Therefore, following \citet{Jaeger1997}'s advice, in the CPS and ACS data which
have fine information about educational attainment, we code people who
drop out of high school in 12th grade as completing high
  school.\footnote{This is automatically implemented in CEPR's public-use CPS
    data. When we augment the CPS data with the ACS data, we recode
    CEPR's ACS education variable so it is consistent with the CPS
    data.}

Early mortality data reports years of
school completed. Beginning in 2003, a share of states began reporting
education in \textit{categories}, where one of the categories is
9th--12th grade, no diploma. By 2018, all deaths are coded using the
2003 categories. It is not possible to disaggregate the
data further. Before 2003, the data only include years of high school
(e.g., 1 year of high school or 4 years of high school). For the data
coded using the pre-2003 categories, we consider dropouts to be people
who attained less than 4 years of high school.

We acknowledge a concern that beginning in 2003, we include people who
attained some 12th grade education as dropouts. We emphasize that, due to
the death certificate aggregation, there is no
other way to harmonize the mortality data over time. However, this data
limiation is unlikely to play a large role in our results. For example, in the 2018
ACS, there are approximately 5 million people who have 12th grade, no
diploma. By contrast, there are 86 million people who do not attain a
high-school degree (and have no 12th grade education at all) and 70
million people who attain a high-school diploma. As a result, the
share of people who receive some 12th grade education but no degree
represents 7\% or less of either group.

\textbf{Moving averages.} Because we use
small population cells (e.g., white high-school dropouts ages 30--34),
to address noise, we use the moving 5-year average for the total population
denominator when 5 years of data are available. In 1993 and 2017, we
use the moving 3-year average (1992--1994 and 2016--2018,
respectively). In 1992 and 2018, we use the predicted values from a
regression of population totals on the adjacent years (1992--1994 and
2016--2018, respectively). 
\begin{itemize}
\item We do not use the 1990 or 1991 CPS data because
the education question changed in 1992, so the estimates of the
dropouts population is discontinuous in 1991. 
\item We do not use the 2019 or 2020 CPS because these extracts are
  not yet harmonized by CEPR. 
\end{itemize} 
We pool annual data into 3-year bins to focus on long-term trends and minimize spurious year-on-year variation in results; e.g. for 1992--94, we use average CPS population in each group and the average number of deaths in each group.

\textbf{Institutionalized populations.} The CPS does not survey
institutionalized populations, e.g. people living in prisons or
hospitals, but deaths in institutions are counted in the mortality
records. To obtain accurate mortality rates, we generate
institutionalized populations in each year as follows:
\begin{enumerate} 
\item We obtain counts of the institutionalized population by age,
  education, gender, race, and year in the
  U.S. Census (1990 and 2000) and American Communities Survey
  (2006--2018). 
\item For years between Censuses (1991--1999) or between the Census
  and American Communities Survey (2001--2005), we impute the
  number of institutionalized people in each
  age-education-gender-race-year by generating a linear prediction of the
  population between nearest surveys. We use the 2006 ACS because it
  has more accurate counts of the institutionalized populations. For example, if there were 1,000
  institutionalized white women aged 50--54 with a high school degree in
  the 1990 Census and 1,200 in the 2000 Census, we would impute 1,100 in
  1995 (where there is no Census available). 
\item We add institutionalized populations to our count of the
  non-institutionalized populations from the CPS. We compute mortality
  rates as the number of deaths divided by the total population. 
\end{enumerate}

In the ages in our sample, this procedure gives that the share of the
white (black) male institutionalized population was
1.3\% (6.2\%) in 2018. The share of the white (black) female
institutionalized population was 0.5\% (0.7\%). In order for errors
from this imputation process to substantially bias
mortality estimates, the institutionalized population would need to
fluctuate non-linearly in the imputed years. Incarceration rates do rise
substantially in the 1990s, but the change is close to linear over
time, suggesting that the imputation is a good approximation.

The 2018 CPS does not include people living in college dormitories. We
do not adjust for this because we only consider the population older
than age 25. 

\textbf{Cause of deaths.} We partition all deaths into five groups:
cancer, heart disease, deaths of despair, injuries, and other
diseases. We construct these groups by using codes from the
International Statistical Classification of Diseases and Related
Health Problems. NCHS reports ICD-9 codes for 1992--1998 and ICD-10
codes for 1999--2018. We list below the codes pertaining to each cause
of death. For consistency, we follow the data appendix and public code
from \citet{Case2017} to define deaths from cancer,
heart disease, and deaths of despair.

\begin{itemize}
\item Cancer. ICD-9: 140--208; ICD-10: C (all). 
\item Heart Disease. ICD-9: 390-429; ICD-10: I0--I9, I11, I13,
  I20--I51. 
\item Deaths of Despair. ICD-9: 571, 850--860, 950--959, 980; ICD-10:
  K70, K73, K74, X40--45, Y10--15, Y45, Y47, Y49, Y87.0. 
\item Injuries. ICD-9: 800-999 \& not a death of despair;
  ICD-10: V, W, X, Y \& not a death of despair. 
\item Other Diseases. All deaths not otherwise classified. 
\end{itemize}

Table~\ref{tab:icd_causes} reports the share of deaths among
25--69-year-olds in 2018, ordered by importance. We report the
categories used in the paper, and then disaggregate remaining deaths
according to major ICD-10 categories.
